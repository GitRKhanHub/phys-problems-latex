\section{8 класс}

\AddProb Что легче: тянуть нагруженную тачку позади себя, либо толкать ее перед собой? Почему?

\AddProb Вася заметил, что тело плавает в мутной жидкости, погрузившись на 2/3 своего объема. 
Если его поместить в воду, то оно плавает, погрузившись наполовину. Чему равна плотность мутной жидкости, если плотность воды 1~г/см$^3$?

\AddProb Трубка, запаянная с одного конца, опускается в жидкость сначала открытым концом вниз, а затем вверх и плавает, находясь в вертикальном положении. 
Длина погруженной в жидкость части трубки в первом случае на $\Delta L = 5$~см больше, чем во втором. 
Найдите высоту $H$ слоя жидкости, зашедшей в трубку в первом случае. Отношение внутреннего сечения трубки $S_1$ к внешнему $S_2$ равно~0.5.

\AddProb На прямой дороге находятся велосипедист, мотоциклист и пешеход между ними. 
В начальный момент времени расстояние от пешехода до велосипедиста в 2 раза меньше, чем до мотоциклиста. 
Велосипедист и мотоциклист начинают двигаться навстречу друг другу со скоростями 20~км/ч и 60~км/ч соответственно. 
В какую сторону и с какой скоростью должен идти пешеход, чтобы одновременно встретиться с велосипедистом и мотоциклистом в месте их встречи?



\section{9 класс}

\AddProb Тело плотностью $\rho$ плавает на границе раздела двух жидкостей с плотностями $\rho_1$ и $\rho_2$. 
Найдите отношение объёма тела, погруженного в нижнюю жидкость $V_1$ к объёму, находящемуся в верхней жидкости~$V_2$.

\AddProb На прямой дороге находятся велосипедист, мотоциклист и пешеход между ними. 
В начальный момент времени расстояние от пешехода до велосипедиста в 2 раза меньше, чем до мотоциклиста. 
Велосипедист и мотоциклист начинают двигаться навстречу друг другу со скоростями 20~км/ч и 60~км/ч соответственно. 
В какую сторону и с какой скоростью должен идти  пешеход, чтобы встретиться с велосипедистом и мотоциклистом в месте их встречи?

\AddProb В сосуде Дьюара хранится $V = 2$~л жидкого азота при температуре $T_1 = 78$~К. За сутки половина этого количества испарилась. 
Определите удельную теплоту испарения азота, если известно, что 40~г льда при температуре $T_2  = 273$~К в том сосуде растает в течение 22.5 часов. 
Скорость подвода тепла внутрь сосуда считать пропорциональной разности температур внутри и снаружи сосуда Дьюара. 
Температура окружающего воздуха $T  = 293$~К, плотность жидкого азота при 78~К составляет 800~кг/м$^3$. 
Удельная теплота плавления льда $\lambda\,=\,334\cdot 10^3$~Дж/кг.

\AddProb Определите силу тока в проводнике $AB$ в электрической цепи, показанной на рисунке. 
Напряжение в цепи $U$ и сопротивление каждого из 10 одинаковых резисторов $R$ считайте известными.



\section{10 класс}

\AddProb Опираясь на барьер катка, мальчик бросил камень горизонтально со скоростью $v_1\,=\,5$~м/с. 
Какова будет скорость $v_2$ камня относительно мальчика, если он бросит камень горизонтально, совершив при броске прежнюю работу, 
но стоя на гладком льду? Масса камня $m\,=\,1$~кг, масса мальчика $M\,=\,50$~кг. Трением о лед пренебречь.

\AddProb Во время сильного снегопада лыжник, бегущий по полю со скоростью $v\,=\,20$~км/ч, заметил, что ему в открытый рот попадает $N_1\,=\,50$ снежинок 
в минуту. Повернув обратно, он обнаружил, что в рот попадает $N_2\,=\,30$ снежинок в минуту. 
Оцените концентрацию снежинок (количество снежинок в единице объема) и дальность прямой видимости в снегопад, 
если площадь рта спортсмена $S\,=\,24$~см$^2$, а размер снежинки $l\,=\,1$~см.

\AddProb В длинной теплоизолированной трубе между двумя одинаковыми поршнями массы $m$ каждый находится 1 моль одноатомного газа при температуре $T_0$. 
В начальный момент скорости поршней направлены в одну сторону и равны $3v$ и $v$. До какой максимальной температуры нагреется газ? 
Поршни тепло не проводят. Массой газа по сравнению с массой поршней пренебречь.

\AddProb Найдите полное сопротивление бесконечной электрической цепи, изображенной на рисунке. Величину $R$ считать известной.



\section{11 класс}

\AddProb На каком минимальном расстоянии $l$ от перекрестка должен начинать тормозить шофер при красном свете светофора, 
если автомобиль движется со скоростью $v\,=\,100$~км/час, а коэффициент трения между шинами и дорогой равен~0.4?

\AddProb Теплоизолированный цилиндр разделен подвижным теплопроводящим поршнем на две части. В одной части цилиндра находится гелий, а в другой - аргон. 
В начальный момент температура гелия равна 300~К, а аргона -- 900~К, и объемы, занимаемые газами, одинаковы. 
Во сколько раз изменится объем, занимаемый гелием, после установления теплового равновесия, если поршень перемещается без трения? 
Теплоемкостью цилиндра и поршня пренебречь. Молярная масса аргона 40~г/моль, молярная масса гелия 4~г/моль.

\AddProb Из вершин правильного шестиугольника со стороной 1~м одновременно пускают по направлению к центру шесть одинаковых заряженных частиц. 
Начальная скорость частиц 1~м/с. Когда расстояние между частицами уменьшилось в два раза, то скорость каждой также уменьшилась вдвое. 
До какого минимального расстояния сблизятся частицы?

\AddProb Одно колено гладкой $U$-образной трубки с круглым внутренним сечением площадью $S$ вертикально, 
а другое наклонено к горизонту под углом~$\alpha$. В трубку налили жидкость плотностью $\rho$ и массой $M$ так, что её уровень в наклонном колене выше, 
чем в вертикальном, которое закрыто лёгким поршнем, соединённым с вертикальной пружиной жёсткостью~$k$. Найдите период малых колебаний этой системы. 
Ускорение свободного падения равно~$g$.