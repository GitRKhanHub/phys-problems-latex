\section{I курс}

\AddProb Утка летела по горизонтальной прямой с постоянной скоростью $u$. 
В неё бросил камень неопытный охотник, причём бросок был сделан без упреждения, 
т. е. в момент броска скорость камня $v$ была направлена точно на утку под углом $\alpha$ к горизонту. 
На какой высоте летела утка, если камень всё же попал в неё?

\AddProb Катушку ниток радиусом $R$ пытаются, прислонив к стене, удержать на весу с помощью собственной нитки, отмотанной на длину $l$. 
При каких значениях коэффициента трения между катушкой и стеной это возможно?

\AddProb Тепловая машина с идеальным газом в качестве рабочего вещества совершает обратимый цикл, 
состоящий из изохоры 1--2, адиабаты 2--3 и изотермы 3--1. 
Найти КПД машины как функцию максимальной $T_2$ и минимальной $T_1$ температур, достигаемых газом в этом цикле.


\section{II и III курсы}

\AddProb Материальная точка массой $m$ начинает двигаться со скорость $v_0$ в вязкой среде, в которой на тело действует сила аэродинамического сопротивления, 
пропорциональная квадрату скорости: $F~=~-\beta\,v^2$, где $\beta$ -- постоянный коэффициент сопротивления, зависящий от формы тела. 
Сила гравитационного притяжения на тело не действует. Найдите зависимость пройденного пути от времени. 
Может ли тело остановиться, если описывать его движение в рамках данной модели?

\AddProb Однородный цилиндрический блок массы $M$ и радиуса $R$ может свободно поворачиваться вокруг горизонтальной оси $O$. 
На блок плотно намотана нить, к свешивающемуся концу которой прикреплен груз $A$. 
Этот груз уравновешивает точечное тело массы $m$, укрепленное на ободе блока, при определенном значении угла $\alpha$. Найти частоту малых колебаний системы.

\AddProb При каком условии амплитуда тока $I$ в цепи, изображенной на рисунке, зависит только от амплитуды приложенного напряжения 
$U~=~U_0\,\cos\,\omega\,t$, но не от его частоты? Найти при этом условии разность фаз между приложенным напряжением и напряжением на концах $RC$-пары.