\section{10 класс}

\AddProb Чему должно быть равно сопротивление $r$ на схеме, чтобы сопротивление между входными клеммами было таким же? Сопротивление $R$ считать известным.

\AddProb В вертикальном цилиндрическом сосуде (тонкой трубке) поверх столба воздуха высотой $h\,=\,50$ см при температуре $T\,=\,27^{\circ}C$ 
помещен столб воды такой же высоты, причем вода залита до открытого края трубки. Какова будет высота столба воды $h_2$ после нагрева системы на 
$\Delta T\,=\,20^{\circ}C$? Атмосферное давление $p_0\,=\,10^5$~Па.

\AddProb Известен следующий способ нахождение центра масс приблизительно однородной палки: концы палки кладут на пальцы, 
которые затем медленно начинают сдвигать (см. рисунок). Когда пальцы соприкоснутся, центр масс палки окажется над ними. Если же, например, 
одна из рук будет в перчатке, а другая нет, то коэффициенты сухого трения в точках касания будут различны: $\mu_1$ и $\mu_2<\mu_1$. 
Можно ли теперь использовать описанный выше способ при определении центра масс палки? Найдите расстояния от пальцев до центра масс как функции 
расстояния $l$ между ними, если изначально палка длиной $L$ положена на пальцы концами. Силу трения считайте не зависящей от скорости движения пальцев.

\AddProb Выдернув чеку, с высоты $H$ роняют осколочную гранату. В некоторый момент времени после неупругого вертикального отскока от земли она взрывается, 
а выделившаяся при взрыве энергия $U$ уходит в кинетическую энергию осколков оболочки. В какой момент времени после отскока взрыв представляет 
наибольшую угрозу для человека, уронившего гранату? На какую максимальную высоту при этом поднимутся осколки, если кинетическая энергия гранаты 
при отскоке от земли изменится в $K$ раз? После взрыва покоящейся гранаты осколки оболочки разлетаются с одинаковыми по модулю скоростями; 
масса оболочки гранаты -- $m$.



\section{11 класс}

\AddProb Чему должно быть равно сопротивление $r$ на схеме, чтобы сопротивление между входными клеммами было таким же? Сопротивление $R$ считать известным.

\AddProb В вертикальном цилиндрическом сосуде (тонкой трубке) поверх столба воздуха высотой $h\,=\,50$ см при температуре $T\,=\,27^{\circ}C$ 
помещен столб воды такой же высоты, причем вода залита до открытого края трубки. Какова будет высота столба воды $h_2$ после нагрева системы на 
$\Delta T\,=\,20^{\circ}C$? Атмосферное давление $p_0\,=\,10^5$~Па.

\AddProb При температуре $T\,=\,27^{\circ}C$ в сосуде содержится разреженный атомарный азот при давлении $P\,=\,1$~кПа. 
Считая, что каждое столкновение атомов азота приводит к их слиянию в молекулу $N_2$, оценить время, за которое $\alpha\,=\,0,01$ массы азота в сосуде 
перейдет из атомарной формы в молекулярную. Радиус атома азота $a\,=\,0.074$ нм. Для оценки среднего значение скорости молекулы 
относительно остальных молекул, используйте среднеквадратичную скорость. Молярная масса азота $\mu\,=\,14$~г/моль.

\AddProb Баскетбольный мяч, отскакивая от пола, налетает на стенку. В момент столкновения со стенкой мяч находился в верхней точке 
своей параболической траектории, его скорость была перпендикулярна стене. При этом мяч поднялся на высоту H и пролетел до стены 
расстояние L. Определить, на какую высоту поднимется мяч, отскочив от стены, и на каком расстоянии от стены он коснется пола. 
Для баскетбольного мяча и покрытий помещения можно считать, что коэффициент сухого трения $\mu>>1$, а все удары происходят 
без остаточной деформации мяча и поверхностей (сопротивление воздуха также несущественно). Подсказка: момент инерции сферической оболочки, 
каковой является мяч, $I\,=\,\frac{2}{3}mR^2$.