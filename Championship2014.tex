\section{I курс}

\AddProb Под каким углом к горизонту следует бросить камень с вершины горы с уклоном $ 45^{\circ}$, 
чтобы он упал на склон на максимальном расстоянии от точки броска?

\AddProb На внутренней поверхности конической воронки с углом $2\alpha$ при вершине на высоте $h$ от вершины находится малое тело. 
Коэффициент трения между телом и поверхностью воронки равен $\mu$. 
Найти минимальную угловую скорость вращения конуса вокруг вертикальной оси $\omega$, при которой тело будет неподвижно в воронке.

\AddProb Моль одноатомного идеального газа нагревают в цилиндре под поршнем, удерживаемым в положении равновесия пружиной, подчиняющейся закону Гука. 
Стенки цилиндра и поршень адиабатические, а дно проводит тепло. Начальный объем газа $V_0$, при котором пружина не деформирована, подобран так, 
что $p_0\,S^2\,=\,k\,V_0$, где $p_0$ -- наружное атмосферное давление, $S$ -- площадь поршня, $k$ -- коэффициент упругости пружины. 
Найти теплоемкость газа для этого процесса.


\section{II и III курсы}

\AddProb Под каким углом к горизонту следует бросить камень с вершины горы с уклоном $ 45^{\circ}$, 
чтобы он упал на склон на максимальном расстоянии от точки броска?

\AddProb Два однородных тонких стержня одинаковой длины и разной массы $m_1$ и $m_2$ лежат на гладкой поверхности параллельно друг другу. 
Стержни могут вращаться без трения вокруг вертикальных неподвижных осей, отстоящих друг от друга на удвоенную длину стержней. 
Свободные концы стержней соединены невесомой пружиной жесткости $k$. Найти период малых колебаний системы.

\AddProb На рисунке изображена схема электромагнитного насоса для перекачки расплавленного металла. 
Участок трубы с расплавленным металлом помещается в магнитное поле, перпендикулярное оси трубы; 
через этот же участок в перпендикулярном (к магнитному полю и оси трубы) направлении пропускается ток. 
Найти избыточное давление $\Delta P$, создаваемое насосом. Величины, обозначенные на рисунке, считать известными.