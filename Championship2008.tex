\section{I курс}

\AddProb Два тяжелых шарика брошены с одинаковыми начальными скоростями из одной точки вертикально вверх, один через $t$ секунд после другого. 
Они встретились в воздухе через $T$ секунд после вылета первого шарика. Определите начальную скорость шариков. Сопротивлением воздуха пренебречь.

\AddProb Подставка массы $m_1$ с полуцилиндрической выемкой радиуса $R$ стоит на гладком столе. 
Тело массы $m_2$ кладут на край выемки и отпускают. Найдите скорость тела и подставки в 
момент, когда тело проходит нижнюю точку полусферы. Трением пренебречь.

\AddProb Сплошной однородный короткий цилиндр радиуса $R$, вращающийся вокруг своей 
геометрической оси со скоростью $n$ об/с, ставят в вертикальном положении на 
горизонтальную поверхность. Сколько оборотов $N$ сделает цилиндр, прежде чем вращение его 
полностью прекратится? Коэффициент трения скольжения между основанием цилиндра и 
поверхностью, на которую он поставлен, не зависит от скорости вращения и равен~$\mu$.


\section{II курс}

\AddProb Однородный тонкий негнущийся стержень массой $m$ поддерживается в горизонтальном 
положении двумя вертикальными опорами у концов стержня. В начальный момент времени $t$~=~0 одна из  
опор выбивается. Найти силу, которая действует на вторую опору сразу же после этого момента. 

\AddProb Для определения отношения удельных теплоемкостей $c_P$ и $c_V$ газа измерили период $T_1$ малых 
колебаний ртути в U-образной стеклянной трубке с незапаянными концами. После этого на обе ветви 
трубки были насажены большие одинаковые полые стеклянные шары с исследуемым газом, вследствие 
чего период колебаний изменился и стал равным $T_2$. Считая процесс сжатия и разрежения газа в шарах 
адиабатическим, вывести формулу для $\gamma = \frac{c_P}{c_V}$. Объем каждого шара равен $V$ [см$^3$], давление газа в них в 
состоянии покоя $h$ [см рт. ст.], а площадь поперечного сечения трубки $S$ [см$^2$]. Объемом незаполненной 
части трубки можно пренебречь по сравнению с объемом шара~$V$.

\AddProb В плоский конденсатор размеров $L\times L\times d$ вдвигается с постоянной скоростью $v$ пластина 
диэлектрика с диэлектрической проницаемостью $\varepsilon$. Определить ток в цепи батареи с ЭДС $\xi$, 
подключенной к конденсатору.


\section{III курс}

\AddProb Материальная точка массой $m$, на которую не воздействуют внешние силы, 
невесомой нитью прикреплена к цилиндру радиусом $R$. Первоначально нить была 
намотана на цилиндр, так что материальная точка касалась цилиндра. В какой-то 
момент времени к массе $m$ приложен импульс силы в радиальном направлении так, что 
нить начала разматываться, начальная скорость массы $m$ составила $u$. Найти закон 
изменения длины смотанной нити.

\AddProb Идеальный одноатомный газ является рабочим веществом в цикле 1-2-3-1, изображенном на
рисунке. Участок 2-3 -- дуга окружности. Найдите КПД такого цикла.

\AddProb В плоский конденсатор размеров $L\times L\times d$ вдвигается с постоянной скоростью $v$ пластина 
диэлектрика с диэлектрической проницаемостью $\varepsilon$. Определить ток в цепи батареи с ЭДС $\xi$, 
подключенной к конденсатору.