\section{10 класс}

\AddProb Вертикально расположенный цилиндрический сосуд, закрытый подвижным поршнем массы $M\,=\,2$~кг, содержит идеальный газ при температуре 
$T_1\,=\,300$~К. На поршень помещают тело массы $m\,=\,100$~г и нагревают газ так, чтобы поршень занял первоначальное положение. 
Найдите температуру $T_2$ нагретого газа. Атмосферное давление не учитывать.

\AddProb Равносторонний треугольник $ABC$ скользит по горизонтальному столу. Известно, что некоторый момент времени точка $A$ имеет скорость 
$v_1\,=\,\sqrt{6}$~м/с, точка $B$ имеет скорость $v_2\,=\,1,5$~м/c, а скорость центра треугольника направлена параллельно стороне~$CB$. 
Какова величина скорости $v_0$ центра треугольника в этот момент времени? 

\AddProb В доску в вершинах правильного шестиугольника вбиты шесть гвоздей. Все гвозди попарно соединены резисторами с сопротивлением~$R$. 
Найдите сопротивление между двумя соседними гвоздями.

\AddProb Легковой автомобиль едет по горизонтальной дороге со скоростью~$v_0$. Если водитель заблокирует задние колеса, 
тормозной путь машины составит $L_1\,=\,28$~м. Если водитель заблокирует передние колеса, тормозной путь будет равен $L_2\,=\,16$~м. 
Каким окажется тормозной путь машины, если заблокировать все четыре колеса? Известно, что центр масс автомобиля расположен на равных расстояниях 
от осей передних и задних колес, диаметр которых одинаков.



\section{11 класс}

\AddProb Тело, свободно падающее с некоторой высоты без начальной скорости, за первую секунду движения проходит путь в $5$ раз меньший, чем за последнюю. 
Найти полное время движения.

\AddProb В системе, изображенной на рисунке, массы грузов равны~$m$, жесткость пружины~$k$. Пружина и нить невесомы, трения нет. 
В начальный момент времени грузы неподвижны, и система находится в равновесии. Затем, удерживая левый груз, смещают правый вниз на расстояние~$a$, 
после чего их отпускают без начальной скорости. Найдите максимальную скорость левого груза в процессе колебаний, 
считая, что нити все время остаются натянутыми, а грузы не ударяются об остальные тела системы.

\AddProb Когда на улице термометр показывает $T_1\,=\,-10^{\circ}C$, а температура батареи отопления $T_0\,=\,55^{\circ}C$, в комнате 
устанавливается температура $T_{K1}\,=\,25^{\circ}C$. Какая температура $T_{K2}$ будет в комнате при том же уровне отопления 
(температура батареи остается неизменной), если наступит похолодание до $T_2\,=\,-30^{\circ}C$?

\AddProb В цепи батарейки и диод идеальные. Ключи разомкнуты, конденсаторы разряжены. Сначала замыкают ключ $K_1$. 
После завершения переходных процессов в цепи замыкают ключ $K_2$. Найдите теплоты $Q_1$ и $Q_2$, выделившиеся на резисторах $R_1$ и $R_2$ с момента 
замыкания ключа $K_1$. Известно, что $\varepsilon_2\,=\,2\varepsilon_1\,=\,2\varepsilon$, $C_1\,=\,C_2\,=\,C$. Заданы только величины $C$ и $\varepsilon$.