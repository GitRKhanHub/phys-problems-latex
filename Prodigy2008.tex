\section{10 класс}

\AddProb Найти сопротивление цепи между точками $A$ и $B$. Сопротивление каждого резистора известно и равно~$R$.

\AddProb Легкий жесткий стержень с шариком массы $m$ на конце свободно вращается в вертикальной плоскости вокруг точки~$O$. 
Известно, что в верхней точке траектории модуль силы натяжения стержня равен $T$ и в два раза меньше, чем в нижней. 
Найдите отношение скоростей шарика в верхней и нижней точках траектории. Ускорение свободного падения равно~$g$.

\AddProb С балкона, находящегося на высоте 20 м, бросают мяч со скоростью 20~м/с. 
Мяч упруго ударяется о стену соседнего дома и падает на землю под балконом. Определите расстояние до соседнего дома, если время полета мяча 1,4~с.

\AddProb В цилиндрическом стакане с водой плавает брусок высоты $L$ и сечения~$S_1$. 
При помощи тонкой спицы брусок медленно опускают на дно стакана. Какая работа при этом совершена? 
Сечение стакана $S_2\,=\,2S_1$, начальная высота воды в стакане тоже $L$, плотность материала бруска $\rho\,=\,0.5\rho_B$, , где $\rho_B$ -- 
известная плотность воды.



\section{11 класс}

\AddProb К концу вертикально висящей пружины длины $l$ прикрепили груз, в результате чего ее длина возросла до~$2l$. 
Предполагая, что удлинение пружины пропорционально нагрузке, найти угловую скорость груза, 
вращающегося на этой пружине по кругу в горизонтальной плоскости, если длина пружины в этом случае~$L$. Массой пружины пренебречь.

\AddProb На горизонтальной поверхности стоят два одинаковых кубика массой~$M$. 
Между кубиками вводится тяжелый клин массой $m$ с углом при вершине~$2\alpha$. Чему равны ускорения кубиков? Трением пренебречь.

\AddProb В плоский конденсатор вдвигается с постоянной скоростью $v$ пластина из диэлектрика. 
Определите ток в цепи батареи, подключенной к конденсатору. Считать известным ЭДС батареи $E$, диэлектрическую проницаемость {\Large $\varepsilon$}, 
высоту пластины $h$, площадь квадратных пластин конденсатора~$S\,=\,b^2$.

\AddProb 4.	Одноатомный газ участвует в некотором процессе. Известно, что его внутренняя энергия пропорциональна квадрату объема. 
Найдите работу, которую совершает газ при сообщении ему количества теплоты~$Q$.
