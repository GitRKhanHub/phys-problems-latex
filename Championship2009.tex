\section{I курс}

\AddProb Два автомобиля движутся к перекрестку по пересекающимся дорогам, образующим 
прямой угол. Скорости автомобилей $v_1$ и $v_2$. Расстояния от автомобилей до перекрестка одинаковы и равны~$L$. 
Каково будет минимальное расстояние между автомобилями в процессе движения?

\AddProb Доска массы $M$ находится на двух одинаковых роликах, вращающихся навстречу друг другу. Расстояние между осями вращения роликов равно~$L$. 
Коэффициент трения между доской и роликом равен~$\mu$. Найдите период колебаний доски.

\AddProb Обруч радиусом $r$ свободно падает в поле силы тяжести, вращаясь вокруг собственной оси, расположенной в горизонтальной плоскости, 
с угловой скоростью $\omega$\,($\omega^2r > g$). Определите радиус кривизны траектории точки обруча, занимающей в данный момент наинизшее положение, 
если скорость падения обруча равна~$v_0$.


\section{II курс}

\AddProb Вася спешит на свидание с Машей, ожидающей его в беседке, находящейся на расстоянии $h$ от аллеи, по которой бежит Вася. 
По какому пути должен двигаться Вася, чтобы достичь желанной цели в кратчайшее время, если его скорость по аллее $v_1$, а по саду $v_2$ ($v_1>v_2$).

\AddProb Металлическая рамка расположена вертикально в однородном магнитном поле, вектор индукции $\vec B$ которого направлен перпендикулярно плоскости рамки. 
Вертикально вниз из состояния покоя начинает движение металлический стержень $AC$. Стержень находится в электрическом контакте с 
вертикальными сторонами рамки, но движется без трения. Определите скорость движения стержня через время $t$ после начала движения. 
Масса стержня $m$, электрическое сопротивление $R$, расстояние между контактами $L$. Электрическим сопротивлением и индуктивностью рамки пренебречь.

\AddProb Небольшое тело скользит по вогнутой поверхности, имеющей параболическую форму (уравнение поверхности $y = ax^2$, где $a$ -- константа). 
Коэффициент трения между телом и поверхностью $\mu\ll 1$. В начальный момент времени координата тела $x_0 = 2.75\mu/a$, а скорость нулевая. 
Какова будет координата тела, когда оно окончательно остановится?


\section{III курс}

\AddProb Внутри гладкой сферы радиуса $R$ находится маленький шарик массы $m$ с зарядом $+q$. Какой заряд $Q$ нужно поместить в нижней точке сферы, 
чтобы шарик удерживался в верхней точке? Поляризацией сферы можно пренебречь.

\AddProb Найти время исчезновения мыльного пузыря радиуса $R_0$, соединенного с атмосферой капилляром, который имеет длину $L$ и радиус канала $r$. 
Поверхностное натяжение $\sigma$, вязкость воздуха~$\eta$.

\AddProb Частица с зарядом $q$ и массой $m$ движется с начальной скоростью $v_0$ в вязкой среде в поперечном магнитном поле с индукцией $\vec B$. 
Сила вязкого трения равна $\vec F = -r\vec v$, где $r$ -- константа. На каком расстоянии от начальной точки частица остановится?