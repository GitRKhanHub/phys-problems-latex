\section{I курс}

\AddProb На гладкой горизонтальной поверхности лежит небольшая шайба массы $m$ и гладкая 
горка массы $M$ высоты $H$. Какую минимальную скорость $v$ надо придать шайбе, чтобы она 
смогла преодолеть барьер?

\AddProb Цепочка массы $M$ и длины $L$ удерживается вертикально, так что ее нижний конец 
касается весов. Цепь отпускают, и она падает на весы. Каковы будут показания весов, когда 
часть цепочки длины $x$ уже упала на весы?

\AddProb Тяжелая горизонтальная плита движется вниз с постоянной скоростью $v_0$. С высоты $h$ 
отпускают без начальной скорости шарик, который начинает подпрыгивать на плите. Считая 
удары шарика о плиту упругими, найти зависимость скорости шарика от времени и построить 
ее график.


\section{II курс}

\AddProb На горизонтальной плоскости лежит капля ртути объемом~$V$. Сверху на 
каплю проставили брусок массой~$M$ так, что капля сильно сплющилась. Оцените 
расстояние~$h$ между бруском и плоскостью, если коэффициент поверхностного 
натяжения ртути $\sigma$ и все поверхности абсолютно не смачиваются ртутью.

\AddProb Цепочка массы $M$ и длины $L$ удерживается вертикально, так что ее нижний конец 
касается весов. Цепь отпускают, и она падает на весы. Каковы будут показания весов, когда 
часть цепочки длины $x$ уже упала на весы?

% Задача 1 Теорема о циркуляции магнитного поля(150)
\AddProb Однородный диэлектрический диск массой~$m$, радиуса~$R$, помещен в однородное магнитное поле индукции~$B$. 
Заряд диска равномерно распределен по его объему и равен~$q$. 
Какую угловую скорость получит диск, если выключить магнитное поле?


\section{III курс}

% Почти задача 1 ЭМИ (152)
\AddProb Прямоугольная рамка со сторонами $a$ и $b$ находиться на расстоянии $L$ от 
прямого провода с током $I$. Какой импульс получит рамка при выключении тока в 
проводе, если активное сопротивление рамки равно $R$, а ее реактивным 
сопротивлением можно пренебречь?

% Задача 1 в Закон сохранения энергии(115)
\AddProb В длинной теплоизолированной трубке между одинаковыми поршнями 
массы $m$ находиться 1~моль одноатомного идеального газа при температуре~$T_0$. 
В начальный момент скорости поршней направлены в одну сторону и равны $v$ и 
$3v$. До какой максимальной температуры нагреется газ? Поршни тепло не 
проводят, массой газа по сравнению с массой поршней пренебречь.

\AddProb Нейтральный пион распался на два $\gamma$-кванта с энергиями $E_1$\,=\,3100~Мэв и $E_2$\,=\,2000~Мэв. 
Найти угол разлета квантов в лабораторной системе отсчета, если энергия покоя пиона $E_{\pi}$\,=\,135~Мэв.